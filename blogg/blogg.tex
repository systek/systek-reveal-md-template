\documentclass[12pt,norsk,a4enc per]{article}
\usepackage[norsk]{babel}
\usepackage[utf8]{inputenc}
\usepackage[T1]{fontenc}
\usepackage{dirtytalk}

\title{Høyne din  kodekvalitet med statisk typing og hissig linting }

\author{Hans Ole Gjerdrum \\Systek AS}
\date{\today}



\begin{document}
\maketitle


\section{Introduction}
Alt for ofte, når jeg bytter prosjekt eller åpner et nytt repo, slår det meg i
ansiktet at koden mest ser ut som et tenåringsrom. Og hvem bryr seg egentlig om
hvordan kodekvaliteten? Du kan jo hevde at det bare er du selv som sitter i din
kode-borg og regjerer allmektig, men da glemmer du de gangene du trenger hjelp,
eller skal par-programere deg gjennom et problem, og den dagen kommer da du skal
forlate prosjektet og gå videre. Da er det koden du etterlater deg som er ditt
\textit{ettermæle}


I nesten alle prosjektene jeg har jobbet som front-end-utvikler har jeg sittet
alene med ansvaret for JavaScript-koden. Mens Javautviklerene lenge har hatt
prosesser og verktøy for å sikre kvaliteten på deres kode, har ikke vi
frontendere jobbet etter slike metoder. Men moderne frontend-kode trenger også
slike metoder.


Og det var kanskje ikke så farlig så
lenge man jobbet i \textit{global.js} på < 500 linjer kode med validering av
inputfelter. Men moderne applikasjoner skal løse uhyre  mer kompliserte oppgaver, de



=> Lesbar, forstålig og forvaltbar kode

\section{Lint}

\subsection{Hva er Linting}
Fra \textbf{Oxford Dictionary} har vi at lint er:
\begin{quote}
Short, fine fibres which separate from the surface of cloth or yarn during
processing.
\end{quote}

\ldots det vi på norsk kaller \textit{lo}. Innen informatikken er linting brukt
om verktøy som skal hjelpe til med å oppdage og å fjerne kildekodens
hybelkaniner, eller sagt på en annen måte \textit{Statisk analyse av kildekode
  for å detektere brudd på definerte regler}. Lint, som verktøy, ble først
utviklet av Stephen C. Johnson hos Bell Labs for sjekk av C-kode. Senere har det
ingått som del av Unix OS.


\subsection{Linting i JavaScript}

Det første verktøyet for å \textit{linte} JavaScript-kode kom i 2002. Det var
online kodesjekkeren \textbf{JSLint}, utviklet av Douglas Crockford. Ved å lime
js-koden sin inn på siden \underline{http://jslint.com/} kunne man få
feilmeldinger og advarsler om hvor koden fravekt et sett med forhåndsdefinerte
regler for hva god kode skulle innebære. For bedre å kunne tilpasse disse
reglene etter egne behov, forket Anton Kovalyov i 2010 JSLint ut til prosjektet
\textbf{JSHint}. JSHint ble også loevert som kommandolinje klient distribuert
som mode-module, at linting kunne bli et steg av kodebyggingen. Dagens
``industristandard'' må sies å være Nicholas C. Zakas' \textbf{ESLint} fra 2013.
Dette verktøyet har tatt brukertilpassingen ennå et steg videre, og har også
muliggjort å anngi hvilke EcmaSript-versjn koden er skrevet i. For utivklere på
TypeScript finnes det også en linte-verktøy kalt \textbf{TSLint} utviklet av
Palantir Technologies.




\section{Type}





\paragraph{Outline}


\section{Results}\label{results}
In this section we describe the results.

\section{Conclusions}\label{conclusions}
We worked hard, and achieved very little.


\end{document}




